%briefly categorize the existing related work 
%describe at least three most relevant related work in details (what they did and findings, how it is similar or different from your work).
In this section, we present the work most related to our study. 
% We divide the prior work into two main areas; work related to the study of API breakage changes and API testing.

\par
Mirhosseini et al. ~\cite{ACM2017_Mirhosseini_AutomatedPullRequests} conducted a study on why developers neglect to update software dependencies, and how effective Greenkeeper and other automated tools are at helping developers keep their dependencies up to date. They found these tools to be useful, with projects that use pull request notifications upgraded on average 1.6x as often as projects that did not use any tools. They also found that, although pull request notifications are useful, developers are often overwhelmed by notifications: only a third of pull requests were actually merged. Through a survey of developers, they found  that one of the primary reasons why practitioners don’t update their dependencies is due to the fear of breaking changes. This demonstrates a need for better tools that give developers higher confidence that updating their dependencies will not break their code.

\par
Xavier et al. \cite{SANER2017_Xavier_HistoricalImpactAnalyisOfAPIBreakingChanges} performed a large-scale study on the historical and impact analysis of API breaking changes in the Java ecosystem. They assess the frequency of breaking changes, the behavior of these changes over time, the impact on clients, and the characteristics of libraries with high frequency of breaking changes. Alongside their results, they provide a set of lessons to better support library and client developers in their maintenance tasks. One of their more interesting results is that, despite their findings that 28\% of all API changes break backwards compatibility, only 2.54\% of clients are potentially impacted.

\par
Brito et al.\cite{ESE2020_Brito_YouBrokeMyCode} conducted a follow up survey to their 2018 paper that introduced \textit{APIDIFF}, a tool to identify API breaking and non-breaking changes between two versions of a Java library \cite{SANER2018_Brito_APIDiff}. After identifying possible breaking changes, they asked the developers to explain the reasons behind their decision to change the APIs, and find that breaking changes are mostly motivated by the need to implement new features, by the desire to make the APIs simpler and with fewer elements, and to improve maintainability. To complement this first study, they conduct an analysis of 110 Stack Overflow posts related to breaking changes. they find that breaking changes have an important impact on clients, since 45\% of the questions are from clients asking how to overcome specific breaking changes.
