%describe the internal and external threats to validity (ref. https://www.verywellmind.com/internal-and-external-validity-4584479)
In this section, we disuses threats to validity that might influence our study.
\subsection{Threats to Internal Validity}
Internal validity concerns factors that could have influenced our analysis and findings. First, we did not perform any filtering of either the client projects or the provider packages we examined, such as removing inactive or "toy" projects. This may have affected our results pertaining how clients respond to the in-range breaking update issue reports, as well as the results involved with the release frequency of providers.
\par
We only collect a single snapshot of each client's dependencies, and therefore we do not have the dependency history of each client project in our data set. In other words, we do not know at what date a client project integrates with each of it's dependencies, or whether they remove any dependencies over their lifetime. We therefore assume that a client uses each of the dependencies for it's lifetime for our analysis in Section \ref{sec:results:rq3}. This may have biased our results when classifying each provider release as either a breaking or non=breaking release based on the ratio of client each released caused an issue report to be created.

\subsection{Threats to External Validity}
Threats to external validity concern the generalization of our technique and findings. First, we only analyse projects that have integrated with the Greenkeeper bot. While Greenkeeper is a popular tool used by many projects \cite{ACM2017_Mirhosseini_AutomatedPullRequests}, they might not be representative of the general population, and while we analyzed a large sample of open-source repositories, these results may not extend to proprietary systems, which may operate under different constraints. Further, Greenkeeper was acquired by Snyk\footnote{https://snyk.io/} in June 2020, and is no longer monitoring projects for in-range breaking updates. Additionally, one of the prerequisites for clients integrating with Greenkeepepr is that the project must have a \textit{package.json} file specifying it's dependencies. This means that Greenkeeper is only able to monitor dependencies in the \textit{npm} ecosystem for in-range breaking updates. Therefore, our analysis only examines npm packages, which can include many new and evolving packages. The dynamic nature of JavaScript can result in harder to detect breaking changes, which may be less of concern in other languages and ecosystems.
\par
Finally, while performing our manual analysis to determine why so many attempts at pinning the dependency being updated did not resolve the client's build, we did not analyze enough samples to be statistically significant. We only analyzed a total of 50 sample, and while they all failed for similar reasons, this is not a statistically significant sample size, and might not lead to generalizable results.