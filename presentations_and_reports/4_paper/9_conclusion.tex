It has become commonplace for developers to reuse code from multiple provider packages, and clients will often specify a range of versions they would like to use from provider packages so that they can automatically receive simple bug fixes and patch updates as the provider releases them. However, these in-range updates can sometimes break a client's build, and since the dependency update is accepted in the client's version specifications, the client package will also not build for any of it's users. In this study, we examined breaking build issue reports created by the Greenkeeper bot for in-range updates to determine characteristics that cause these build failures, as well as how client’s respond to these issues and how they resolve  their build. We found that the majority of issues are created for patch updates, and that packages with higher total and more frequent releases tend to cause more breakages. Often, the client simply updating their dependency specifications is enough for them to resolve their build, and we found no indication that releases that break a relatively high proportion of client's builds prompts a response in the provider packages.
