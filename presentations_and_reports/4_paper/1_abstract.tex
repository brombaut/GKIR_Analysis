\begin{abstract}
Software dependency relationships allow a client package to reuse certain versions of a provider package. These provider packages often release versions containing bug fixes, new functionalities, and security enhancements, so it is important for the clients to keep their dependencies up to date. Client's can specify a range of versions from providers they would like to accept, usually including small upgrades and fixes, which will automatically be installed whenever the client's project is built. However, these in-range updates can sometimes break a client's build, which is problematic as users of the client package will also automatically receive the in-range update and not be able to build the client's project. To understand the characteristics of these in-range breaking updates and how they are resolved, we examine in-range breaking build issue reports opened by the Greenkeeper bot and actions that are taken by clients to get their build passing again. We find that the majority of issues are created for patch updates, and that packages with higher total and more frequent releases tend to cause more breakages. Often, the client simply updating their dependency specifications is enough for them to resolve their build, and we found no indication that releases that break a relatively high proportion of client's builds prompts a response in the provider packages.
\par
Source-code Link: \url{https://github.com/brombaut/LOG6307_Project} 
\end{abstract}